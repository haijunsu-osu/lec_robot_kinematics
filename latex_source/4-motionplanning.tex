\section{Motion Planning}

Consider a start position $P_S$ and end position $P_E$ of a rigid body. To move from $P_S$ to $P_E$, a trajectory must be generated, hence the term ``motion planning'' or ``trajectory generation''.  Such a trajectory is defined by the path it takes to get between positions (the Interpolated Motion) and the mathematical function describing the trajectory between positions (the Trajectory Planning).

\subsection{Interpolated Motion}

Any motion the robotic arm makes use of interpolation to navigate between positions $P_S$ and $P_E$. Such interpolation can be done in several ways. Two common methods of interpolation are:
\begin{itemize}
\item \textbf{Joint motion moveJ():} Interpolation using 6DOF representation
\item \textbf{Linear motion moveL():} Interpolation using equivalent angle axis representation
\end{itemize}

\subsubsection{Joint motion}

Using the joint values $\theta_S$ and $\theta_E$ a joint motion, using each joint independently of one another, can be made. The idea is to have each joint of a robotic arm follow their seperate trajectories between two points.

$$\text{moveJ}(t) = \begin{bmatrix} \theta_{1}(t) & ... & \theta_{n}(t) \end{bmatrix}^T$$

Using such a joint motion to go from $P_S$ to $P_E$ requires that these positions can be represented using the joints of the robotic arm. This can be done by using Inverse Kinematics on said positions converting them into joint values.

$$\theta_S = \begin{bmatrix} \theta_{1,S} & ... & \theta_{n,S} \end{bmatrix}^T = \text{IK} \begin{pmatrix} P_S\end{pmatrix} \quad \theta_E = \begin{bmatrix} \theta_{1,E} & ... & \theta_{n,E} \end{bmatrix}^T = \text{IK} \begin{pmatrix} P_E \end{pmatrix}$$

\subsubsection{Linear motion}

Provided the position $P_S$ and $P_E$ in transform matrix format ${^B_S T}$ and ${^B_E T}$, it becomes possible to make a single transform matrix able to represent a linear motion between the two points ${^S_E T} = {^B_S T}^{-1} \cdot {^B_E T}$. And in cases where only joint values are provided, Forward Kinematics can be used to get said positions.

$$P_S = \text{FK} \begin{pmatrix} \begin{bmatrix} \theta_{1,S} & ... & \theta_{n,S} \end{bmatrix}^T \end{pmatrix} \quad P_E = \text{FK} \begin{pmatrix} \begin{bmatrix} \theta_{1,E} & ... & \theta_{n,E} \end{bmatrix}^T \end{pmatrix}$$

Using ${^S_E T}$, a linear motion between $P_S$ and $P_E$ becomes plausible using an angle-axis representation defined as:

$$\text{moveL}(t) = \begin{bmatrix} x(t) & y(t) & z(t) & \theta(t)\end{bmatrix}^T$$

Where 

$$\begin{matrix} x(t) = {^S_E T_x}, & y(t) = {^S_E T_y}, & z(t) = {^S_E T_z}, & \theta(t) = \frac{cos^{-1} ({^S_E T_{11}} + {^S_E T_{22}} + {^S_E T_{33}} - 1)}{2} \end{matrix}$$

Do note that a rotation matrix representation of $\theta$ can be made using the below equation:

$${^S_E R} = \begin{bmatrix} k_x \cdot k_x \cdot (1 - \text{cos}\theta) + \text{cos}\theta & k_x \cdot k_y \cdot (1 - \text{cos}\theta) - k_z \cdot \text{sin}\theta & k_x \cdot k_z \cdot (1 - \text{cos}\theta) + k_y \cdot \text{sin}\theta \\ k_y \cdot k_x \cdot (1 - \text{cos}\theta) + k_z \cdot \text{sin}\theta & k_y \cdot k_y \cdot (1 - \text{cos}\theta) + \text{cos}\theta & k_y \cdot k_z \cdot (1 - \text{cos}\theta) - k_x \cdot \text{sin}\theta \\ k_z \cdot k_x \cdot (1 - \text{cos}\theta) - k_y \cdot \text{sin}\theta & k_z \cdot k_y \cdot (1 - \text{cos}\theta) + k_x \cdot \text{sin}\theta & k_z \cdot k_z \cdot (1 - \text{cos}\theta) + \text{cos}\theta \end{bmatrix}$$

Where

$$K = \begin{bmatrix} k_x \\ k_y \\ k_z \end{bmatrix} = \frac{1}{2 \cdot \text{sin}(\theta)} \cdot \begin{bmatrix} {^S_E T_{32}} - {^S_E T_{23}} \\ {^S_E T_{13}} - {^S_E T_{31}} \\ {^S_E T_{21}} - {^S_E T_{12}} \end{bmatrix}$$

\subsection{Trajectory Planning}
The trajectory planner of any robotic arm can be set to run in a meriat of ways. In this repository, two such trajectory planners have been used.
\begin{itemize}
\item \textbf{Cubic Polynomial}
\item \textbf{Parabolic Blend} aka. Linear Segments with Parabolic Blend (LSPB)
\end{itemize}

Note that for any joint or DOF over time $\theta(t)$, movement between 2 positions may be described using these 4 constraints:

\begin{itemize}
\item $\theta(0) = \theta_0$
\item $\theta(f) = \theta_f$
\item $\dot{\theta}(0) = \dot{\theta}_0$
\item $\dot{\theta}(f) = \dot{\theta}_f$
\end{itemize}

\subsubsection{Cubic Polynomial}

The 4 contraints described above may be satisfied using a cubic polynomial such as the one seen below:

$$\begin{matrix} \theta(t) = a_0 + a_1 \cdot t + a_2 \cdot t^2 + a_3 \cdot t^3 , t \in [0, t_f] \end{matrix}$$

Using this cubic polynomial, the velocity of the trajectory can be found as its derivarive $\dot{\theta}(t)$ and the acceleration as the second derivative $\ddot{\theta}(t)$.

$$\begin{matrix} \dot{\theta}(t) = a_1 + 2 a_2 \cdot t + 3 a_3 \cdot t^2 ,  t \in [0, t_f] \\ \ddot{\theta}(t) = 2 a_2 + 6 a_3 \cdot t , t \in [0, t_f] \end{matrix}$$

Using $\theta(t)$ and $\dot{\theta}(t)$, the 4 constraints can be expressed as the following equations:

\begin{itemize}
\item $\theta(0) = a_0$
\item $\theta(f) = a_0 + a_1 \cdot t_f + a_2 \cdot t_f^2 + a_3 \cdot t_f^3$
\item $\dot{\theta}(0) = a_1$
\item $\dot{\theta}(f) = a_1 + 2 a_2 \cdot t + 3 a_3 \cdot t^2$
\end{itemize}

Solving for the parameters $a_0$, $a_1$, $a_2$, and $a_3$ using the above 4 expressions, the following equations are obtained:

\begin{itemize}
\item $a_0 = \theta(0)$
\item $a_1 = \dot{\theta}(0)$
\item $a_2 = \frac{3}{t_f^2} \cdot (\theta_f - \theta_0) - \frac{2}{t_f} \cdot \dot{\theta} - \frac{1}{t_f} \cdot \dot{{\theta}}_f$
\item $a_3 = -\frac{2}{t_f^3} \cdot (\theta_f - \theta_0) + \frac{1}{t_f^2} \cdot (\dot{\theta}_0 + \dot{\theta}_f)$
\end{itemize}
Fig. \ref{fig:cubic} show the plots of the cubic polynomial, its first derivative (velocity), and its second derivative (acceleration). Assuming the robotic arm has $n$ joints, a matrix-vector equation can be established to describe the cubic polynomial of each of these joints:

$$\text{moveJ}(t) = \begin{bmatrix} a_{1,0} & a_{1,1} & a_{1,2} & a_{1,3} \\ ... & ... & ... & ... \\ a_{n,0} & a_{n,1} & a_{n,2} & a_{n,3} \\ \end{bmatrix} \cdot \begin{bmatrix} 1 & t & t^2 & t^3 \end{bmatrix}^T , \text{for n joints} $$

\begin{figure}[htb]
\centering
\subfigure{\includegraphics[width=0.25\textwidth]{imgs/CubicP.jpg}}
\subfigure{\includegraphics[width=0.25\textwidth]{imgs/dCubicP.jpg}}
\subfigure{\includegraphics[width=0.25\textwidth]{imgs/ddCubicP.jpg}}
\caption{Cubic polynomial, its derivative, and second derivative}
\label{fig:cubic}
\end{figure}


\subsubsection{Linear Segments with Parabolic Blend (LSPB)}

The parabolic blend is derived using the 3 Kinematics Equations in terms of rotation $\theta(t)$

$$\begin{matrix} \theta(t) = \frac{1}{2} \ddot{\theta} \cdot t^2 + \dot{\theta}_0 \cdot t + \theta_0 \\ \dot{\theta}(t) = \ddot{\theta} \cdot t + \dot{\theta}_0 \\ \dot{\theta}(t)^2 = \dot{\theta}^2_0 + 2 \ddot{\theta} \cdot (\theta_f - \theta_0) \end{matrix}$$

Using the 4 constraints mentioned above, these 3 Kinematics Equations wind up like this:

$$\begin{matrix} \theta(t) = \frac{1}{2} \ddot{\theta} \cdot t^2 + \theta_0 \\ \dot{\theta}(t) = \ddot{\theta} \cdot t \\ \dot{\theta}(t)^2 = 2 \ddot{\theta} \cdot (\theta_f - \theta_0) \end{matrix}$$

Using these 3 equations, the 3 segments of the Parabolic Blend can be defined as:

$$\theta(t) = \begin{Bmatrix} \begin{matrix} \theta_0 + \frac{1}{2} \ddot{\theta} \cdot t^2, & \text{if } t_0 \leq t \leq t_b \\ \theta_b + a \cdot t_b \cdot (t - t_b), & \text{if } t_b < t < t_f - t_b \\ \theta_f - \frac{1}{2} \ddot{\theta} \cdot t^2 + \theta_0, & \text{if } t_f - t_b \leq t \leq t_f \end{matrix} \end{Bmatrix} , \text{for } \begin{bmatrix} t_0, t_f \end{bmatrix}$$

Where

$$t_b = \frac{t_f}{2} - \text{abs} \begin{pmatrix} \sqrt{ \frac{\ddot{\theta}^2 \cdot t_f^2 - 4 \cdot \ddot{\theta} \cdot (Pe-Ps) }{(2 \cdot \ddot{\theta})} } \end{pmatrix}$$

The parabolic blend comes with its own condition, because it has to comply with the 3 Kinematic Equations. Using the third Kinematic Equation, this condition can be derived assuming no overlap occurs between the 3 segments meaning, $t_f \geq 2 t_b$ must be true:

$$ \dot{\theta}(t_b)^2 = 2 \ddot{\theta} \cdot (\theta_f - \theta_0) \to \dot{\theta}(t_f)^2 \geq 4 \ddot{\theta} \cdot (\theta_f - \theta_0) \to (\ddot{\theta} \cdot t_f)^2 \geq 4 \ddot{\theta} \cdot (\theta_f - \theta_0) \to \ddot{\theta} \geq \frac{4 \cdot (\theta_f - \theta_0)}{t_f^2} $$

See Fig. \ref{fig:parabolic} for a graphical representation of the Parabolic Blend trajectory, its velocity, and its acceleration.

\begin{figure}[htb]
\centering
\includegraphics[width=0.5\textwidth]{imgs/ParabolicBlend.png}
\caption{Parabolic blend trajectory}
\label{fig:parabolic}
\end{figure}

\subsection{Calculating execution time $t_f$}

The used method of calculating execution time $t_f$ of any trajectory makes use of the assumption that a parabolic blend is used. 

Consider that the Parabolic Blend has a maximum velocity $v_{max}$ and maximum acceleration ${a_{max}}$. This means the Parabolic Blend has a velocity of $v_{max}$ during its linear segment and an acceleration of $a_{max}$ during acceleration (negative $a_{max}$ during deceleration). This way, the time $t_b$ becomes equal to:

$$ t_b = v_{max} / a_{max} $$

Using $t_b$, it becomes possible to determine the execution time $t_f$. In cases where $2 t_b = t_f$, $t_f$ becomes:

$$ \begin{matrix} t_f = \sqrt{ \frac{4 \cdot (\theta_f - \theta_0)}{a_{max}} } , \text{if } 2 t_b \geq t_f \end{matrix} $$

Note how even though the expression $2 t_b > t_f$ is theoretically implausible, it accounts for cases when the specified $v_{max}$ surpases the maximum reached velocity of the specified Parabolic Blend.

On the other hand, when $2 t_b < t_f$, the linear segment has to be taken into account. This is done by using the below expression:

$$ \begin{matrix} t_f = t_b + \frac{\theta_f - \theta_0}{v_{max}} , \text{if } 2 t_b < t_f \end{matrix} $$

Using the two above equation describing execution time $t_f$, the theoretically lowest $t_f$ is acquired assuming the specified $v_{max}$ and $a_{max}$ apply to the trajectory. That is to say these equations assume a parabolic blend is used as the trajectory planner.
