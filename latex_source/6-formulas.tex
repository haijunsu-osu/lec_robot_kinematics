\section{Formula Cheat Sheet}

\begin{minipage}[c]{0.4\textwidth}
Cross product:\quad
$\mathbf{a} \times \mathbf{b}=\left[\begin{array}{l}
a_{2} b_{3}-a_{3} b_{2} \\
a_{3} b_{1}-a_{1} b_{3} \\
a_{1} b_{2}-a_{2} b_{1}
\end{array}\right]$
\end{minipage}
\hfill
\begin{minipage}[c]{0.12\textwidth}
Trigonometric Identities:
\end{minipage}
\hfill
\begin{minipage}[c]{0.4\textwidth}
$$\begin{aligned}
&\sin(\alpha)^2 + \cos(\alpha)^2 = 1 \\
&\sin (\alpha\pm\beta)=\sin \alpha \cos \beta\pm\cos \alpha \sin \beta \\
&\cos (\alpha\pm\beta)=\cos \alpha \cos \beta\mp\sin \alpha \sin \beta \\
\end{aligned}$$
\end{minipage}
Determinant:

\begin{minipage}[c]{0.25\textwidth}
$$
|A|=\left|\begin{array}{ll}
a & b \\
c & d
\end{array}\right|=a d-b c .
$$
\end{minipage}
\hfill
\begin{minipage}[c]{0.7\textwidth}
$$
\begin{aligned}
|A|=\left|\begin{array}{lll}
a & b & c \\
d & e & f \\
g & h & i
\end{array}\right| &=a\left|\begin{array}{ll}
e & f \\
h & i
\end{array}\right|-b\left|\begin{array}{cc}
d & f \\
g & i
\end{array}\right|+c\left|\begin{array}{cc}
d & e \\
g & h
\end{array}\right| \\
&=a e i+b f g+c d h-c e g-b d i-a f h
\end{aligned}
$$
\end{minipage}
A manipulator may have special configurations, called ``isotropic points'', that are characterized by the Jacobi matrix having orthogonal columns of equal length, thus $J^{T} J=\delta I$ for some $\delta \in \mathbb{R}$.

\subsection{Denavit-Hartenberg Parameters}

\begin{minipage}[c]{0.1\textwidth}
DH-table (link-index $i$):
\end{minipage}
\hfill
\begin{minipage}[l]{0.25\textwidth}
\begin{tabular}{ |c|c|c|c|c| } 
\hline
$i$ & $a_{i-1}$ & $\alpha_{i-1}$ & $d_i$ & $\theta_i$ \\
\hline
1 & ... & ... & ... & ... \\ 
\hline
... & ... & ... & ... & ... \\ 
\hline
\end{tabular}
\end{minipage}
\hfill
\begin{minipage}[c]{0.6\textwidth}
\begin{center}
\begin{enumerate}
	\item Shift $Z_{i-1}$ by $a_{i-1}$ along $X_{i-1}$.
	\item Rot.\ $Z_{i-1}$ by $\alpha_{i-1}$ about $X_{i-1}$ \& shift by $d_i$ along $Z_i$.
	\item Rot.\ $X_{i-1}$ by $\theta_i$ about $Z_{i}$ \& move it to $Z_i$.
\end{enumerate}
\end{center}
\end{minipage}
\begin{center}
	\includegraphics[width=7cm]{imgs/2_dh_params.png}
	\includegraphics[width=9cm]{imgs/11.png}
\end{center}
Homogeneous transformation from link $i$ to $i-1$:\quad	${ }^{i-1}_iT=\left[\begin{array}{ccc|c}
c \theta_{i} & -s \theta_{i} & 0 & a_{i-1} \\
s \theta_{i}\ c \alpha_{i-1} & c \theta_{i}\ c \alpha_{i-1} & -s \alpha_{i-1} & - s \alpha_{i-1}\ d_{i} \\
s \theta_{i}\ s \alpha_{i-1} & c \theta_{i}\ s \alpha_{i-1} & c \alpha_{i-1} & d_{i}\ c \alpha_{i-1} \\
\hline 0 & 0 & 0 & 1
\end{array}\right]$.\\
Inverse of the homogeneous transform:\quad	${ }^{A}_BT^{-1}={ }^{B}_AT=\left[\begin{array}{ccc|c}
 & {}^{A}_BR^T & & -{}^{A}_BR^T {}^AP_{Borg} \\
\hline 0 & 0 & 0 & 1
\end{array}\right]$.

\subsection{Jacobian}

\textit{Singularity}: the end-effector locally looses at least 1 DOF. This happens, when $Z$-axes are aligned $\Leftrightarrow$ the Jacobian does not have full rank $\Leftrightarrow det(J)=0$. Small end-effector motions require large joint motions near singularities.

\subsubsection{Velocity Propagation}

Linear and angular velocities at joint $i+1$ (with scalar $\dot{d}_{i+1}$ or $\dot{\Theta}_{i+1}$ for prismatic/ revolute joints):
$$
\begin{aligned}
	&\ {}^{i+1} \omega_{i+1} = {}^{i+1}_{i} R \cdot {}^{i} \omega_{i} + \dot{\Theta}_{i+1} \cdot {}^{i+1} \hat{Z}_{i+1}\\
	&\ {}^{i+1} v_{i+1} = {}^{i+1}_{i} R ({}^{i} v_{i} + {}^{i} \omega_{i} \times {}^{i} P_{i+1}) + \dot{d}_{i+1} \cdot {}^{i+1} \hat{Z}_{i+1}\\
\end{aligned}
$$
Then read off the Jacobian from:\quad\quad $\left[\begin{array}{c}
	\dot{x}_P \\
	\dot{x}_R \\
	\end{array}\right]_{6 \times 1}=\left[\begin{array}{l}
J_{x_P} \\
J_{x_R}
\end{array}\right]_{6 \times n} \mathbf{\dot{\Theta}}=J_{6 \times n}\left[\begin{array}{c}
	\dot{\Theta}_{1} \\
	... \\
	\dot{\Theta}_{n}
	\end{array}\right]_{n \times 1}
	$.

\subsubsection{Force/ Torque Propagation}

\begin{minipage}[c]{0.35\textwidth}
Force $f$ and torque $n$ at joint $i$:
\end{minipage}
\hfill
\begin{minipage}[c]{0.6\textwidth}
$\begin{aligned}
	&{ }^{i} f_{i}={ }_{i+1}^{i} R \cdot{ }^{i+1} f_{i+1} \\
	&{ }^{i} n_{i}={ }_{i+1}^{i} R \cdot{ }^{i+1} n_{i+1}+{ }^{i} P_{i+1} \times{ }^{i} f_{i}
	\end{aligned}
$
\end{minipage}\\
\begin{minipage}[c]{0.3\textwidth}
The force/ torque exerted on the $i$th joint:
\end{minipage}
\hfill
\begin{minipage}[c]{0.3\textwidth}
\begin{center}
	$
\tau_{i}={ }^{i} f_{i}^{\mathrm{T} i} Z_{i}={ }^{i} f_{i}^{\mathrm{T}}\left(\begin{array}{l}
0 \\
0 \\
1
\end{array}\right)
$
\end{center}
\end{minipage}
\hfill
\begin{minipage}[c]{0.3\textwidth}
\begin{center}
$
\tau_{i}={ }^{i} n_{i}^{\mathrm{T} i} Z_{i}={ }^{i} n_{i}^{\mathrm{T}}\left(\begin{array}{l}
0 \\
0 \\
1
\end{array}\right)
$
\end{center}
\end{minipage}\\
Then read off the Jacobian from:\quad\quad $\left(\begin{array}{c}
\tau_{1} \\
\tau_{2} \\
\vdots \\
\tau_{n}
\end{array}\right)=\tau={ }^{A} J^{T A} \mathcal{F}={ }^{A} J^{T}\left(\begin{array}{c}
{}^{A}f \\
{}^{A}n
\end{array}\right)
$,\\
where $\mathcal{F}$ is a $6\times 1$ force-torque vector in frame $\{A\}$.

\subsubsection{Explicit Form}

Jacobian in frame $\{0\}$:\quad\quad\quad\quad\quad ${ }^{0} J=\left[\begin{array}{cccc}
\frac{\partial}{\partial q_{1}}\left({ }^{0} x_{P}\right) & \frac{\partial}{\partial q_{2}}\left({ }^{0} x_{P}\right) & \cdots & \frac{\partial}{\partial q_{n}}\left({ }^{0} x_{P}\right) \\
\bar\epsilon_{1} \cdot\left({ }_{1}^{0} R \cdot Z\right) & \bar\epsilon_{2} \cdot\left({ }_{2}^{0} R \cdot Z\right) & \cdots & \bar\epsilon_{n} \cdot\left({ }_{n}^{0} R \cdot Z\right]
\end{array}\right)$\\
with ${ }^{0} Z_{i}={ }_{i}^{0} R{ }^{i} Z_{i} ; \quad{ }^{i} Z_{i}=Z=\left[\begin{array}{l}0 \\0 \\1\end{array}\right]$. The indicator variable $\bar\epsilon_{i}$ is $1$ if joint $i$ is revolute, otherwise it is $0$. Then, rotate ${}^0J$ into the required reference frame:

\begin{center}
	${}^A J (\Theta ) = \begin{pmatrix} {}^A_B R & \mathbf{0} \\ \mathbf{0} & {}^A_B R \end{pmatrix} {}^B J (\Theta)$
\end{center}

\subsection{Solutions to Common Trigonometric Equations}
\label{subsec:trig_equations}

To solve equations of the form $A \cos \theta + B \sin \theta = C$, use the identity:

\[
\sqrt{A^2 + B^2} \cos(\theta - \phi) = C
\]

where $\tan \phi = B/A$.

Then, $\theta = \phi + \arccos\left( \frac{C}{\sqrt{A^2 + B^2}} \right)$ or $\theta = \phi - \arccos\left( \frac{C}{\sqrt{A^2 + B^2}} \right)$.

For the PUMA 560, this is used to solve for $\theta_3$.





