\section{Appendices}

%This section contains supplementary material, derivations, tables, or additional notes relevant to the main chapters. Add your appendices content here.

\subsection{Euler Angles}
\label{subsec:euler_angles_appendices}

Euler angles are used to describe the orientation of a rigid body with respect to a fixed coordinate system. There are 12 sets of Euler angles (extrinsic) and 12 sets of fixed (intrinsic) Euler angles, each corresponding to different rotation sequences.

% Figure: 12 sets of Euler angles (extrinsic)
\begin{figure}[h]
    \centering
    \subfloat[Extrinsic Euler angles set 1]{\includegraphics[width=0.4\textwidth]{imgs/Euler_1.png}}
    \hfill
    \subfloat[Extrinsic Euler angles set 2]{\includegraphics[width=0.4\textwidth]{imgs/Euler_2.png}}
    \caption{12 sets of Euler angles (extrinsic)}
\end{figure}

% Figure: 12 sets of fixed (intrinsic) Euler angles
\begin{figure}[h]
    \centering
    {\includegraphics[width=0.4\textwidth]{imgs/Euler_fixed_1.png}}
    \hfill
    {\includegraphics[width=0.4\textwidth]{imgs/Euler_fixed_2.png}}
    \caption{12 sets of fixed (intrinsic) Euler angles}
\end{figure}

\newpage

\subsection{Algebraic Formula}
\begin{minipage}[c]{0.4\textwidth}
Cross product:\quad
$\mathbf{a} \times \mathbf{b}=\left[\begin{array}{l}
a_{2} b_{3}-a_{3} b_{2} \\
a_{3} b_{1}-a_{1} b_{3} \\
a_{1} b_{2}-a_{2} b_{1}
\end{array}\right]$
\end{minipage}
\hfill
\begin{minipage}[c]{0.12\textwidth}
Trigonometric Identities:
\end{minipage}
\hfill
\begin{minipage}[c]{0.4\textwidth}
$$\begin{aligned}
&\sin(\alpha)^2 + \cos(\alpha)^2 = 1 \\
&\sin (\alpha\pm\beta)=\sin \alpha \cos \beta\pm\cos \alpha \sin \beta \\
&\cos (\alpha\pm\beta)=\cos \alpha \cos \beta\mp\sin \alpha \sin \beta \\
\end{aligned}$$
\end{minipage}
Determinant:

\begin{minipage}[c]{0.25\textwidth}
$$
|A|=\left|\begin{array}{ll}
a & b \\
c & d
\end{array}\right|=a d-b c .
$$
\end{minipage}
\hfill
\begin{minipage}[c]{0.7\textwidth}
$$
\begin{aligned}
|A|=\left|\begin{array}{lll}
a & b & c \\
d & e & f \\
g & h & i
\end{array}\right| &=a\left|\begin{array}{ll}
e & f \\
h & i
\end{array}\right|-b\left|\begin{array}{cc}
d & f \\
g & i
\end{array}\right|+c\left|\begin{array}{cc}
d & e \\
g & h
\end{array}\right| \\
&=a e i+b f g+c d h-c e g-b d i-a f h
\end{aligned}
$$
\end{minipage}

To compute atan2(a, b), use the following definition:
\[ \operatorname{atan2}(a, b)= \begin{cases}\arctan \left(\frac{a}{b}\right) & \text { if } b>0 \\ \frac{\pi}{2} & \text { if } b=0, a>0 \\ \text { undefined } & \text { if } b=0, a=0 \\ -\frac{\pi}{2} & \text { if } b=0, a<0 \\ \arctan \left(\frac{a}{b}\right)+\pi & \text { if } b<0\end{cases} \]

\subsection{Solutions to Common Trigonometric Equations}
The following formula presents solutions to some commonly used algebraic (trigonometric) equations in robotics, as compiled from Craig's book~\cite{Craig2005}.

\begin{center}
    %\centering
    \includegraphics[width=0.75\textwidth]{imgs/TrigEq_solution.png}
    %\caption{Solutions to commonly used trigonometric equations (from Craig's book \cite{Craig2005}).}
\end{center}
